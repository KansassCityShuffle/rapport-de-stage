\chapter*{Annexe 2}
\label{annexe:2}
\addcontentsline{toc}{chapter}{Annexe 2}

%changer le format des sections, subsections pour apparaittre sans le num de chapitre
\makeatletter
\renewcommand{\thesection}{\@arabic\c@section}
\makeatother

%recommencer la numérotation des section à "1"
\setcounter{section}{0}

Cette annexe détaille les types de données énumérés dans le document de Spécification Technique de Besoin Logiciel qui ont été évoqués plus en amont dans ce rapport, section \nameref{sec:qualite-interne}. 
On rappelle que la présentation de ces types est motivée par l'idée d'exposer une démarche documentaire sous son aspect formel --le contenu des données elles-mêmes ou leur signification n'ayant pas de réel intérêt ici. 
Notons qu'à chaque fois qu'un champ issu d'une structure de données correspond également à un type structuré, on déroulera en cascade la définition jusqu'à ce que tout puisse être décrit avec des types de base. 

\section{Message OCCUPANCY\_GRID}

Topic: \path{/map} \\
Type: \path{nav_msgs/OccupancyGrid} \\
Emetteur: \path{/hector_mapping} \\
Destinataire: \path{/map_bridge} \\
Fréquence d’émission moyenne : $0,5$ Hz \\
Description: Emis par SLAM, donne la probabilité d’occupation de la grille représentant la carte \\
Contenu:

\begin{figure}[h]
    \begin{tabular}{|l|l|l|l|}
      \hline
      \textbf{Contenu} & \path{header} & \path{info} & \path{data} \\
      \hline
      \textbf{Type} & \path{std_msgs/Header} & \path{std_msgs/MapMetaData} & \path{int8[]}\\
      \hline
    \end{tabular}
  \caption{Contenu du message Occupancy\_Grid}
\end{figure}

\subsection{Header}
\begin{figure}[h]
    \begin{tabular}{|l|l|l|l|}
      \hline
      \textbf{Contenu} & \path{seq} & \path{stamp} & \path{frame_id} \\
      \hline
      \textbf{Type} & \path{uint32} & \path{time} & \path{string}\\
      \hline
    \end{tabular}
  \caption{Contenu du message Header}
\end{figure}

\subsection{Info}
\begin{figure}[h]
    \begin{tabular}{|l|l|l|l|l|l|}
      \hline
      \textbf{Contenu} & \path{map_load_time} & \path{resolution} & \path{width} & \path{height} & \path{origin}\\
      \hline
      \textbf{Type} & \path{time} & \path{float32} & \path{uint32} & \path{uint32} & \path{geometry_msgs/Pose}\\
      \hline
    \end{tabular}
  \caption{Contenu du message Info}
\end{figure}

\subsection{Origin}
\begin{figure}[h]
    \begin{tabular}{|l|l|l|}
      \hline
      \textbf{Contenu} & \path{position} & \path{orientation} \\
      \hline
      \textbf{Type} & \path{geometry_msgs/Point} & \path{geometry_msgs/Quaternion} \\
      \hline
    \end{tabular}
  \caption{Contenu du message Origin}
\end{figure}

\subsection{Position}
\begin{figure}[h]
    \begin{tabular}{|l|l|l|l|}
      \hline
      \textbf{Contenu} & \path{x} & \path{y} & \path{z}\\
      \hline
      \textbf{Type} & \path{float64} & \path{float64} & \path{float64}\\
      \hline
    \end{tabular}
  \caption{Contenu du message Position}
\end{figure}

\subsection{Orientation}
\begin{figure}[h]
    \begin{tabular}{|l|l|l|l|l|}
      \hline
      \textbf{Contenu} & \path{x} & \path{y} & \path{z} & \path{w}\\
      \hline
      \textbf{Type} & \path{float64} & \path{float64} & \path{float64} & \path{float64}\\
      \hline
    \end{tabular}
  \caption{Contenu du message Orientation}
\end{figure}

\subsection{Data}

Représente une carte à 2 dimensions où chaque cellule contient sa probabilité d’occupation. 
Les lignes ont la priorité dans l’indexage de la carte.
La carte commence à la position $(0,0)$.
Les probabilités sont comprises dans $[0,100]$.
-1 représente une case non explorée.