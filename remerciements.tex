\chapter*{Remerciements}

En prélude à ce rapport, je tiens à remercier M. David Daumand, ingénieur logiciel au sein de la société \gls{SII}, qui a défini, tutoré et défendu ce projet. 
Merci à lui pour le suivi et l'implication sans faille sur lesquels j'ai pu compter durant ces six mois, et ce malgré les efforts organisationnels qui lui incombaient.
J'ai tout particulièrement apprécié la passion pour les sujets de robotique et de développement qu'il a su transmettre avec patience et surtout, avec un enthousiasme constant.
Ce stage se concluant par une embauche, je ne pourrais que souligner la part importante qu'il a jouée dans le cheminement menant à cette situation et les conseils expérimentés qu'il a pu prodiguer à cet effet. 

Je salue sincèrement M. Adel Hafiane, enseignant-chercheur et responsable du laboratoire de vision par ordinateur de l'INSA Centre Val-de-Loire, qui m'a assistée en qualité d'enseignant-référent.
Je le remercie d'avoir veillé assidûment sur ce projet, tant dans les conditions matérielles de son déroulement, que dans son orientation technique et pédagogique au regard de ma formation. 
J'exprime toute ma gratitude à son égard pour nous avoir, Alban Chazot et moi, orientés vers \gls{SII} pour nos stages de fin d'étude avec la confiance qui était la sienne. 

Enfin je remercie l'ensemble de l'équipe qui m'a accueillie avec bienveillance, sympathie et humanité au sein de l'agence de \gls{SII} Bourges. 
Merci aux consultants, stagiaires, Responsable de site et collaborateurs 
qui ont jalonné mon quotidien et enrichi mon expérience professionnelle nouvelle dans ses aspects techniques, organisationnels et bien sûr humains. 