\begin{otherlanguage}{british}
  \begin{abstract}

This document deals about achievements conducted for \gls{SII} --the ``Company for Industrial Information Technology''-- as a part of an internship in the engineering school INSA Centre Val-de-Loire.  
It completes the Security and Information Technologies school programme and lies under the aegis of a partnership between \gls{SII} and INSA Centre Val-de-Loire. 
My work in the company is placed under \emph{SII Research} organization. 
This entity intends to build innovating offers as well as communicating around technologically advanced topics through proof of concept projects.
Also called demonstrators, theese achievements are financed with SII own resouces and must intrinsically address a major group of clients. 
For their parts, my studies and tasks are directed to Aero-Space Defense sector through the following problem. \\
In order to carry out specific tasks, robotic systems must rely on consitent environment perception which can be distinguished into different types.
On one hand we consider external environment like spatial features or obstacles and, on the second hand, the location of the mobile platform itself in this external environment.   
These complementary aspects are better known as \gls{SLAM}, for Simultaneous Localization And Mapping. 
\gls{SLAM} algorithms implementation typically requires an external measurement system such as a \gls{LIDAR}, for LIght Detection And Ranging, and internal wheels encoders able to restitute odometry records. 
This report adresses these issues through the specification, the conception and the implementation of software components that intend to control a mobile robot and compute his environment perception in real time. 
The project described here covers the whole system, from physical acquisition devices to results presentation through a human-machine interface.
It relies on the mobile robot \emph{Wifibot Lab} equipped with the \emph{RPLidar A2} \gls{LIDAR} for spatial distances acquisition. 
Computations are centralized on a user workstation, where part of processing is handled by a \gls{ROS} (Robot Operating System) network.  
Presentation and control interface, for its part, is written in C++ with the Qt Library. 
 
 
    \begin{center}
      \textbf{Mots-clés}   
      
      Cartographie et Localisation Simultannées
      
      Développement logiciel
      
      Temps-réel
      
      Système embarqué
      
      Robot Operating System
    \end{center}

      
 \end{abstract}

\end{otherlanguage}
