\chapter*{Annexe 1}
\label{annexe:laserscan}
\addcontentsline{toc}{chapter}{Annexe 1}

%changer le format des sections, subsections pour apparaittre sans le num de chapitre
\makeatletter
\renewcommand{\thesection}{\@arabic\c@section}
\makeatother

%recommencer la numérotation des section à "1"
\setcounter{section}{0}

Cette annexe donne la définition du message \path{nav_msgs/LaserScan.msg}. 
Ce type de message est fourni en sortie du n\oe{}ud \path{/rplidarNode} par le biais du topic \path{/scan} présenté figure \ref{fig:rosnet}.
Le type \path{Header} encapsulé est également défini. 

\section{Définition du message nav\_msgs/LaserScan.msg}

\begin{lstlisting}[style = custombash]
# Single scan from a planar laser range-finder
#
# If you have another ranging device with different behavior (e.g. a sonar
# array), please find or create a different message, since applications
# will make fairly laser-specific assumptions about this data

Header header            # timestamp in the header is the acquisition time 
                         # of the first ray in the scan.
                         # in frame frame_id, angles are measured around 
                         # the positive Z axis (counterclockwise, if Z is up)
                         # with zero angle being forward along the x axis
                         
float32 angle_min        # start angle of the scan [rad]
float32 angle_max        # end angle of the scan [rad]
float32 angle_increment  # angular distance between measurements [rad]

float32 time_increment   # time between measurements [seconds] - if your
                         # scanner is moving, this will be used in 
                         # inetrpolating position of 3d points
float32 scan_time        # time between scans [seconds]

float32 range_min        # minimum range value [m]
float32 range_max        # maximum range value [m]

float32[] ranges         # range data [m] (Note: values < range_min or > range_max should be discarded)
float32[] intensities    # intensity data [device-specific units].  If your
                         # device does not provide intensities, please leave
                         # the array empty.
\end{lstlisting}

\section{Définition du message std\_msgs/Header.msg}

\begin{lstlisting}[style = custombash]
# Standard metadata for higher-level stamped data types.
# This is generally used to communicate timestamped data 
# in a particular coordinate frame.
# 
# sequence ID: consecutively increasing ID 
uint32 seq
#Two-integer timestamp that is expressed as:
# * stamp.sec: seconds (stamp_secs) since epoch (in Python the variable is called 'secs')
# * stamp.nsec: nanoseconds since stamp_secs (in Python the variable is called 'nsecs')
# time-handling sugar is provided by the client library
time stamp
#Frame this data is associated with
# 0: no frame
# 1: global frame
string frame_id
\end{lstlisting}