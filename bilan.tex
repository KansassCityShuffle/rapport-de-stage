\chapter{Bilan et perspectives}
\label{chap:bilan}

\section{Organisation du travail}
\label{sec:orga}
  \subsection{Application du référentiel qualité interne}
  
  Nous exposons ici les éléments fourni par SII ayant guidé de manière significative la conduite du projet. 
  Ceux-ci seront plus ou moins détaillés en fonction du temps consacré à leur mise en place.
  
  Les premiers éléments oraganisationnels qui ont été formalisés s'incarnent par deux documents de spécification appelés \gls{STBL} et \gls{DCL}.
  Le premier vise, comme son nom l'indique, à exprimer le plus objectivement possible le besoin qui justifie la réalisation logicielle, notamment au moyen d'exigences représentées figure \ref{fig:exigences}. 
  Cette énumération haut-niveau permet d'appréhender simplement et rapidement le périmètre du logiciel ciblé. 
  Cependant, la réalisation de ce document en bonne et dûe forme nécessite une liste exaustive d'exigences fonctionnelles vues comme des ensembles de fonctions qu'il convient de décrire en terme de rôle, d'entrées, de sorties et
  de traitements. 
  Dans cadre de ce stage, nous définissons des fonctions comprises dans des modules distincts et étant reliées de la manière suivante : 
  
  \textbf{Ici il manque le schéma de relation entre fonctions que j'avais fait pour le STBL}
  
  Puis, pour chacune des fonctions, on va définir des sous-fonctions, idéalement jusqu'à un niveau de granularité maximal où tous les types de données sont énumérés. 
  Par exemple, on établit que la fonction \emph{Transmettre} du module \textbf{\textcolor{red-stbl}{Map\_bridge}} regroupe les quatre sous-fonctions \emph{Recevoir un message}, \emph{Formater un message} et \emph{\'{E}mettre un message}. 
  Afin d'illustrer cette démarche, nous donnons ici pour exemple la spécification de la fonction \emph{Transmettre : Formater un message}. 
  
  \textbf{1. Rôles } \\
  Cette fonction est activée à l’arrivée d’une demande de traitement de message sur la liaison SLAM $\Longleftrightarrow{}$ MAP\_BRIDGE.
  Elle intervient après que la fonction de réception ait attesté de la consistance du message reçu.
  Elle permet d’extraire la charge utile du message reçu et d’en réduire significativement le volume avant envoi.  Les données en sorties sont sérialisées.
  Cette extraction s’effectue en comparant OCCUPANCY\_GRID avec une copie locale de la dernière carte reçue OLD\_OCCUPANCY\_GRID.
  
  \textbf{2. Entrées }
  \begin{figure}[!h]
    \begin{center}
      \begin{tabular}{|l|l|}
	\hline
	\textbf{Désignation} & \textbf{Type de données} \\
	\hline
	Message OCCUPANCY\_GRID & OccupancyGrid \\
	\hline
      \end{tabular}
    \end{center}
    \caption{Entrées de la fonction \emph{Transmettre : Formater un message}}
  \end{figure}
  
  \textbf{3. Sorties }
  \begin{figure}[!h]
    \begin{center}
      \begin{tabular}{|l|l|}
	\hline
	\textbf{Désignation} & \textbf{Type de données} \\
	\hline
	Message MAP & OccupancyGridUpdate \\
	Message INI\_MAP & IniOccupancyGrid \\
	\hline
      \end{tabular}
    \end{center}
    \caption{Sorties de la fonction \emph{Transmettre : Formater un message}}
  \end{figure}
  
  Notons que la spécification des types de données d'entrées-sorties est donné dans un document externe. Ceux relatifs à l'exemple ci-dessus sont décrits en \textbf{Annexe XX}.  
  
  \textbf{4. Traitements }
  
  \textbf{NB} : Les traitements sont idéntifiés de manière unique au sein d'un document, ils satisfont également des règles permettant d'être traités automatiquement par des logiciels de suivi de projet. 
  Nous donnons ici un exemple succint permettant d'en saisir le sens. Dans la pratique une sous-fonction donne lieu à plusieurs traitements.
  
  \textbf{$[$REQ\_STBL\_MAP\_BRIDGE\_FORM\_1$]$}
  
    \hspace{10mm} Si OLD\_OCCUPANCY\_GRID existe : 
    
    \hspace{10mm} On compare le champ ``data'' de OCCUPANCY\_GRID et de OLD\_OCCUPANCY\_GRID. 
    
    \hspace{10mm} On crée la structure MAP à partir des valeurs de ``data`` qui diffèrent. 
    
    \hspace{10mm} MAP est une chaîne de caractères dont les valeurs sont séparées par des '', ``.\\ 
  \textbf{$[$FIN\_REQ$]$}
  
  Le Document de Conception Logiciel est quant à lui intervenu plus tard dans l'avancée du projet. 
  Il consiste ne la formalisation des éléments conceptuels, en terme d'architecture physique et logicielle du projet. 
  Ces briques ayant été largement explicitées au long de ce rapport nous n'attayerons pas d'avantage ce point. 
  
  Par ailleurs, un diagramme de Gantt présentant une granularité hebdomadaire a été effectué au mois de mars, ce document est présenté figure \textbf{X}.
  Cette réalisation vise à définir et ordonnancer les tâches à réaliser et, d'autre part, à estimer l'impact temporel de chacune d'entre-elles. 
  Ce document a été réalisé dans une optique d'organisation personnelle mais a également constitué un outil d'évaluation et de communication avec M. Daumand qui a été en grande partie affecté en missions hors de l'agence. 
  Il s'agissait donc pour nous de fixer les jalons clés, les \emph{dead-lines} et les versions du logiciel à produire afin qu'il suive de manière pragmatique l'avancée du travail. 
  
  \subsection{Vers une conduite AGILE adaptée}
  
  Au référentiel interne de qualité du logiciel se sont ajoutées à la gestion de ce projet certaines bonnes pratiques issues de la formation en Architecure et Sécurité du Logiciel prodiguée à l'INSA Centre Val-de-Loire
  qui ont pu être exploitées dès lors qu'Alban Chazot et moi-même avons été amenés à travailler de concert.
  Cette phase est intervenue dès que nos fonctionnalités respectives, décrites dans le cadre de nos sujets de stage étaient opérationnelles.  
  En particulier, nous avons utilisé le gestionnaire de versions Git au travers du système de gestion de dépôts (forge) GitLab auquel nous avons appliqué une stratégie de création de branches appelée ''Git Flow``.
  Ce modèle d'utilisation de Git vise à minimiser les conflits et les régressions, accentuer la visibilité des tâches en cours ou terminées et à scinder clairement les phases de développement du logiciel. 
  \`{A} cet effet nous avons adopté la démarche illustrée sur la figure \ref{fig:gitflow} qui se lit de bas en haut et où les commits sont représentés par des points.
  Cela consiste à travailler sur une branche de développement (\path{develop}), à partir de laquelle nous créons une branche par fonctionnalité (\path{feature}), éventuellement une branche \path{hotfix} pour une correction de bug
  et, enfin, une branche \path{realease} qui va aboutir sur la dernière version stable du logiciel.
  Cette branche est ensuite répercutée sur \path{develop} et \path{master}, dans le premier cas  pour continuer les actions de développement et dans le second, afin de permettre la récupération des sources ou son déploiement. 
  
  \begin{figure}
    \centering
      \includegraphics[width=.7\linewidth]{figures/gitflow}  
    \captionof{figure}{Politique de gestion de branches Git adoptée}
    \label{fig:gitflow}
  \end{figure}
  
  Aussi, l'\path{Apploader} a constitué une refonte majeure de l'architecture du logiciel qui a été conceptualisée et développée en équipe. 
  Afin de répartir les tâches que nécessitait son implémentation, nous avons pu expérimenter l'utilisation d'un \gls{Kanban} recensant : 
  
  \begin{itemize}
   \item les tâches à effectuer, par exemple \emph{Enregister les données de cartographie dans le format DOG}
   \item la compléxité relative de chacune d'entre-elles 
   \item les dépendances de certaines tâches les unes par rapport aux autres
  \end{itemize}

  Définir la ''complexité`` d'une tâche revient à lui attribuer un score en se référant aux scores attribués pour les tâches précédentes. 
  Les méthodes AGILE --notamment SCRUM-- préconnisent le recours à des échelles de quantification relatives plutôt qu'à des jours hommes ou d'autres échelles se voulant précises. 
  Concrètement on peut utiliser les valeurs de la suite de Fibonacci qui suivent ce que l'on appelle la courbe d'incertitude, à savoir qu'au plus une valeur est élevée, au plus l'écart avec la valeur suivante sera grande. 
  Nous avons estimé la complexité des unités de travail à réaliser selon une méthode également empruntée à SCRUM, appelée \emph{planning poker}. 
  Tous les participants disposent d'un jeu de carte qui, dans notre cas, comporte les nombres $\dfrac{1}2, 1, 2, 3, 5, 8, 13, \infty$. 
  L'effort de réalisation est ensuite estimé en même temps pour une tâche donnée, puis soumis à discussion dans le cas de désaccord. 
  Cette pratique est à la fois rapide, ludique et présente l'intérêt de dénuer d'emblée la quantification de l'influence mutuelle des participants. 
  
  Ce Kanban a ainsi été adopté dans une optique d'ordonnancer les tâches et, accessoirement, de les répartir entre les membres de l'équipe. 
  Il a aussi permis d'attester visuellement de nos avancées respectives et du nombre de tâches en cours permettant un allègement ou une répartition de la charge si besoin.
  La consultation du nombre de tâches restantes a également joué dans nos décisions de poursuivre telle ou telle fonctionnalité au profit d'autres par exemple. 
  
\section{Résultats en vue d'un prolongement}  
  \subsection{Un module de SLAM en adéquation avec les attentes du projet}
  
  Nous nous intéressons dans un premier lieu à la fidélité des résultats de SLAM et dans un second temps, à une appréciation de l'ensemble du projet, incluant également les travaux d'Alban Chazot et bien sûr la vision de M. Daumand.  
  
  La figure suivante donne un apperçu de la précision des résultats de SLAM. 
  Elle met en parallèle les résultats cartographiques du système avec un plan d'évacuation des locaux dans lesquels l'acquisition a pu être menée. 
  Pour des raisons pratiques, toutes les pièces n'ont pu être scannées, puisque celles-ci hébergent des collaborateurs ou directeur d'agence menant leurs activités. 
  
  \textbf{Ici il manque une cartographie riche des locaux + un schéma qui reprend le plan d'évacuation pour les comparer. Manque aussi un paragraphe qui discute des résultats et des limites d'Hector SLAM et le résultat de 
  la projection des objets détectés sur la carte (sous forme de screen shot où m'on voit la carto, un objet détecté sur le flux vidéo)}
  
  Enfin, la réussite du projet dans sa globalité a pu être attestée par les résultats d'une présentation devant des collaborateurs commerciaux et directeurs de projet qui est intervenue le 13 juin 2017.
  Cette étape a permis de recceuillir des avis extérieurs sur le projet lui-même et d'estimer sa potentielle viabilité commerciale et technique. 
  Ayant été positivement perçu, le système démonstrateur réalisé à été présenté brièvement auprès de Nexter System, enclin à organiser une rencontre à cet effet au mois de septembre. 
  
  \subsection{Pourquoi continuer SRT2M}
  
  La présentation du projet à un client tel que Nexter Systems étant un objectif établi des les premières phases du projet, nous avons veillé à maximiser le potentiel d'appropriation de ce dernier par de tierces personnes.
  Cette volonté s'est incarnée par la mise en place des éléments suivants : 
  
  \begin{itemize}
   \item un manuel utilisateur interactif et ergonomique présenté au sein d'une interface web
   \item un manuel d'installation, agrémenté de résolutions problèmes pouvant être rencontrés lors de cette phase
   \item la création d'une documentation automatique et, là aussi, interactive grâce au logiciel Doxygen
   \item l'automatisation de l'installation pour la distribution Debian 8, notamment au moyen du gestionnaire de paquets apt, et la gestion des dépendances manquantes sur la station hôte
   \item l'adoption de ROS répondant à un critère de haute flexibilité fonctionnelle et matérielle
   \item l'adaptation des possibilités de l'IHM au caractère modulaire des éléments du middleware ROS 
  \end{itemize}
  
  \textbf{Ici j'ai envie de souligner l'intérêt de continuer le projet, en rappelant les diverses idées dont nous avons pu discuter : système distribué sur des véhicules autonomes avec des drônes etc... 
  Peut-être qu'il est intéressant d'ajouter les images du scénario qu'on a présenté à Marc et Patrice ? }
  
\section{Retour d'expérience}
  \subsection{Apports et difficultés du projet}
  
  Ce projet s'est trouvé complexe et stimulant à tous les niveaux. 
  
  Les phases de spécification du besoin ont été fastidieuses tant du fait de la rigueur des documents requis qu'au regard de mon manque de vision intial et de ma difficulté à me représenter ce que serait le système en bout de stage. 
  Cela s'explique notamment par une méconnaissance intiale du domaine de la robotique et du secteur de la défense.
  Les aides conjointes de MM. Daumand et Hafiane, de vastes recherches documentaires et un intérêt certain pour ce sujet où tout était à construire ont fort heureusement motivé une rapide appropriation des problématiques et enjeux afférents. 
  
  Les outils utilisés tels que ROS et à plus faible mesure Qt, ou les fondements théoriques du SLAM puis d'Hector SLAM ont également nécessité des temps de compréhension et d'assimilation conséquents. 
  Bien que n'ayant pas outrepassé le temps imparti à la recherche de solutions techniques, il s'est avéré difficile de passer plusieures semaines à approfondir des connaissances très spécifiques --comme dans le cas de ROS-- sans 
  avoir la moindre idée de ce que donneront les débuts de l'implémentation. 
  Cet effet tunnel s'est trouvé renforcé du fait des délais d'acquisition du matériel et des incertitudes liées au fonctionnement du robot. 
  
  Aussi, nous avons eu la chance d'appréhender la réalisation d'un système à naître, sans qu'aucune étude préalable n'ai été menée. 
  Ce point a à la fois constitué un atout majeur dans le choix de ce stage, mais également un point d'incertitude indéniable qui complique la visualisation d'un système final. 
  Ces difficultés ont peut-être avant tout été surmontées par la communication au sein de l'équipe, permettant de désamorcer les situations de doutes. 
  
  Enfin, j'ai particulièrement apprécié l'alternance entre une période assez longue de travail individuel, et un travail en équipe --certes restreinte-- dans une seconde phase du développement. 
  Dans le premier cas, cela force à une organisation, une prise de décisions et de responsabilités individuelles accrues qui n'ont que très peu été expérimentées lors de projets scolaires au long-cours.
  Le deuxième temps de développement avec Alban Chazot a permi l'expérimentation d'outils issus de méthodes AGILES et adaptés à notre équipe minimale.  
  Cette deuxième phase s'est faite sur un ton plus détendu, les fonctionnalités primordiales du système ayant été présentées et validées. 
  
  \subsection{\'{E}volution personnelle au sein de la structure}
  
  Bien qu'elle soit inclue dans un groupe international, l'agence SII Bourges affiche un esprit \emph{start-up} indéniable. 
  L'équipe sur place est à taille humaine, d'une moyenne d'âge située autours d'une trentaine d'années et présente de normes décontractées soutenues par une hiérarchie horizontale.
  La société SII dénote d'une envie de séduire les plus jeunes actifs en breuvetant le slogan \#FUNgenieur qui vise à dépoussiérer l'image du geek austère, en proposant salles de jeu, de sport ou de détente dans ses agences et, 
  finalement, en misant sur un management de proximité accru.
  \`{A} cet effet, Mme Aurélie Merlin, Campus Manager rattachée à l'agence Île-de-France encadre la communauté de stagiaires et participe à la faire vivre par le biais de défis hauts en couleurs : concours photo, présentation 
  des stages en trois minutes, après-midi dédiés aux échanges entre stagiaires. 
  Cette culture d'entreprise atypique fait de SII une entreprise favorable à une intégration réussie, permettant d'y envisager facilement un début de carrière professionnelle. 
  
  Souhaitant exercer dans le domaine de la sécurité informatique, et mettre rapidement à profit la formation reçue à l'INSA Centre Val-de-Loire, j'ai accepté un poste de consultante en cyber-sécurité au sein de SII.
  
  \textbf{Manque une description un peu plus précise du poste}