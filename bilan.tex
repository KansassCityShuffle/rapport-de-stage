\chapter{Bilan et perspectives}
\label{chap:bilan}

\section{Organisation du travail}
\label{sec:orga}
  \subsection{Application du référetiel qualité interne}
  
  \subsection{Vers une conduite AGILE adaptée}
  
\section{Des résultats en vue d'un prolongement}  
\section{Des résultats qui laissent envisager une suite}
  \subsection{SRT2M à la fin du stage}
  
  Résultats : présentation devant des collaborateurs commerciaux et directeurs de projet qui va se prolonger sur une présentation client au mois de septembre 
  
  Ce que l'on a mis en place pour que le projet se prolonge : 
  
  modularité, MAJ etc
  
  potentiel d'apropriation du projet par de tierces personnes facilité : manuel utilisateur en interface web, troubleshooting, documentation automatisée 
  
  
  
  \subsection{Pourquoi prolonger le projet ?}
  
\section{Retour d'expérience}
  \subsection{Apports et difficultés du projet}
  
  L'adoption des normes de spécification et de conception du logiciel interne a constitué une phase intéressante et complexe à la fois. 
  En effet, la partie Design et Conception de l'IHM, figure \ref{fig:exigences} témoigne d'a prioris au sujet de la conception du logiciel. 
  La DCC003 évoque des ``modules'' --à savoir des exécutables indépendants de l'IHM-- dont l'état devrait apparaitre dans un panneau de contrôle. 
  Or, supposer de la présence de plusieurs exécutables en amont de la conception est un manquement aux préconisations de formalisation du besoin logiciel. 
  Cette partie qui incombe habituellement au futur propriétaire du logiciel a été réalisée en même temps que la conception, d'avantage dans l'optique d'une découverte des normes de spécification que dans celle de réellement 
  s'appuyer sur les documents produits. Le travail s'en est trouvé d'une part simplifié, puisqu'il m'était facile de décrypter un document que j'avais moi-même écrit, et d'autre part complexifié sur le plan du formalisme
  des documents à produire. Cela a principalement engendré des écarts entre les documents de spécification et le code source assez importants, tant et si bien que les parties plus poussées de la documentation en sont 
  rapidement devenues obsolètes. 
  
  Temps d'appropriation du sujet et du domaine qui constitue une découverte, tout autant que les outils utilisés (ROS) ou que les fondements théoriques découverts (SLAM). 
  
  \subsection{\'{E}volution personnelle au sein de la structure}
  
  CDI, sécurité etc... 