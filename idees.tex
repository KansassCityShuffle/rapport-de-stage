Objectifs du stage : 
1 - Présentation du contexte (SII , historique, secteur d'activité - mettre l'accent sur BU ASD - CA ASD - secteur porteur pour la société) 
2 - Définition des objectifs du stage, des moyens mis à disposition

Fonctionnalités : 

1 - SLAM 
2 - Vue première personne
3 - Rejeu de sauvegardes : définition d'un modèle de document XML, encodage des données les plus lourdes (trajectoire et carte)
4 - Lissage du rendu des obstacles : 	définition d'une structure de données permettant d'agglomérer les points détectés par le laser, 
					politique d'agglomération, définition de la structure de rendu
5 - Intégration joystic virtuel
6 - Contrôle du robot : envoi des commandes de contrôles et remontées des informations odométriques		
7 - Visualisation sur l'interface graphique
8 - Gestion des paramètres de l'application : accent sur la volonté de produire un logiciel modulaire 
9 - Mise en place d'un système de journalisation flexible à sorties multiples 
10 - Visionnage des données du LIDAR 
11 - Transmission WiFi, un choix fonctionnel au détriment de la sécurité (script d'exécution en shell non interactif) 

Architecture et choix techniques :

1 - Périphériques matériels : robot, LIDAR et Raspberry PI 3 modèle B, schéma de mise en oeuvre liaision Wifi et poste
de travail utilisateur 
1 - Architecture de l'IHM : utilisation de Qt, signaux et slots 
2 - ROS : architecture (rqt-graph), topics, frames, focus sur les noeuds personnellement implémentés  

Bilan sur la qualité du logiciel : 

1 - La gestion des ressources
  1 - Analyse mémoire du programme de l'IHM  
  1.1 - Application de l'outil d'analyse mémoire ``Memcheck'' depuis l'interface graphique de Qt Creator. 
  Description de l'outil 
  Configuration 
  Exécutions interne et externe à l'IHM 
  Contextes d'analyse : En dernier lieu, nous devons élaborer des scénarios pertinents et suffisamment exhaustifs voués à couvrir une majeure partie des contextes 
  d'exécution qui pourraient être amenés par un utilisateur. Ces sénarios, qui sont présentés en Annexe X, ont tous été passés au crible de l'outil d'analyse.   

  1.2 - Interprétation des problèmes détectées

  2 - Application de correctifs 
  2.1 - Smarts Pointers  
  2.2 - Fonctions de dé-allocation appropriées ( free() / delete / delete[] )   

  3 - Résultats 
  3.1 - Utilisation de la mémoire 
  3.2 - Utilisation CPU
  3.3 - Erreurs externes 
  De nombreuses erreurs externes sont recensées suite à l'analyse mémoire effectuée.
  Celles-ci sont dues aux librairies utilisées par l'application et peuvent être classées en diverses catégories en fonction de l'emplacement de ces dernières. 
  On peut principalement citer les erreurs ayant pour source les librairies de calcul propres à NVidia (libGLX_nvidia.so.X), celles propres à QT ( ) ou encore celles .  