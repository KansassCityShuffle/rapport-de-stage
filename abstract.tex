\begin{abstract}
    Pour répondre aux besoins qui motivent leur réalisation, les systèmes robotiques mobiles doivent s'appuyer sur une connaissance plus ou moins fiable de leur environnement. 
    On distingue la perception de l'environnement externe du robot de la perception que le robot a de lui-même --en tant que plateforme mobile-- au sein de cet environnement.
    Ces deux versants, qui tiennent une place prépondérante dans le domaine des véhicules autonomes ou intelligents, 
    sont connus sous le nom de SLAM (Localisation et Cartographie Simultannées). 
    Ce rapport aborde ces questions au travers de la spécification, de la conception et de l'implémentation de briques logicielles destinées à contrôler un robot mobile 
    et rendre état de son environnement externe tout comme de sa localisation en temps réel.   
    Le projet décrit dans ce document recouvre le système complet, allant des périphériques physiques d'acquisition de données jusqu'à la présentation des résultats au travers d'une 
    IHM interactive. 

    In order to carry out specific tasks, robotic systems must rely on a consitent environment perception. 
    Two types of environment perception can be distinguish. The first is about external environment, and the second is about the location of the mobile plateform in this environment.   
    These two complementary aspects are known as SLAM (Simultaneous Localisation And Mapping). 
    This report adresses these issues through the specificaiton, the conception and the implementation of software components. 

  \begin{center}
    \textbf{Mots-clés}
    
    Cartographie et Localisation Simultannées
    
    Développement logiciel
    
    Temps-réel
    
    Système embarqué
    
    Robot Operating System
  
  \end{center}
\end{abstract}
