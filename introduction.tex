\chapter*{Introduction}
Issu d’un partenariat entre \gls{SII} et l’INSA Centre Val-de-Loire, ce stage vise à la réalisation d’une application logicielle au sein de la société SII, sur le site de Bourges. 
Il s’inscrit dans le domaine du développement informatique appliqué à la robotique. 
Les briques logicielles implémentées devront permettre de cartographier l’environnement d’un robot mobile, ainsi que de le localiser en quasi-temps réel. 
Ces deux versants s’appuient prioritairement sur des mesures spatiales, forunies par un \gls{LIDAR} embarqué sur le robot. 

Ce stage s'inscrit dans un périmètre plus large qu’est la mise en \oe{}uvre d’un Système Robotique Tactique Multi-Missions pour la surveillance et l’aide à la prise de décisions dans les milieux à risques (\gls{SRT2M}), 
adressé au secteur de la Défense. 
Il permet à un opérateur d’effectuer des missions de contrôle, de surveillance et de recherche en terrain dangereux ou potentiellement dangereux directement depuis un ``shelter'', à savoir une zone abritée. 
L’opérateur dispose d’une flotte de robots terrestres et aériens qu’il peut téléguider à distance. 
\`{A} mesure que l’exploration progresse, l’opérateur visualise d’une part les données cartographiques et la localisation des véhicules au sein de cette carte, et d’autre part, 
des flux vidéo émanant de caméras à 360\degre également embarquées. 
Ces flux vidéo sont soumis à des traitements qui permettent de détecter des objets d’intérêts et de les classifier de manière automatique. 
C'est Alban Chazot, étudiant ingénieur à l'INSA Centre Val-de-Loire qui a assuré, pendant son stage, la réalisation de cette fonctionnalité.   
Les classifications sont ensuite incrustées sur la carte générée, de manière à ce que l'utilisateur puisse visionner la classe, la position et la taille des entités détectées en une seule et même zone de rendu. 

%En définitive, nous disposons d’un système capable de naviguer en autonomie et de qualifier précisément les « objets » qu’il rencontre, typiquement : une personne, un véhicule, un bâtiment, une arme…
Le système ainsi mis en place ouvre le champ à divers scénarios d’utilisation. 
Ceux-ci trouvent leur pertinence dès lors que le recours à des opérateurs humains sur le terrain concerné présente un risque ou de fortes contraintes. 
Un scénario possible est l’exploration d’environnements accidentés (incendies, incidents nucléaires), où la localisation de victimes (USART) et l’estimation des dégâts sont réalisables de manière sécurisée. 

Annonce du plan. 