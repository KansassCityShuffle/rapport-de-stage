\makeglossary

\newglossaryentry{SII}
{
  name={SII},
    description={Société pour l'Informatique Industrielle}
}

\newglossaryentry{ROS}
{
  name={ROS},
    description={Robot Operating System}
}

\newglossaryentry{LIDAR}
{
  name={LIDAR},
    description={LIght Detection And Ranging est un dispositif physique muni d'un ou de multiples faisceaux lumineux ainsi que de capteurs permettant la mesure de distances spatiales entre le matériel émetteur et les obstacles rencontrés par les ondes émises.
    \textbf{Ajouter un peu de théorie sur la triangulation et particulièrement, sur la technique de mesure employée par le RPLIDAR A2} }
}

\newglossaryentry{RADAR}
{
  name={RADAR},
    description={RAdar Detection And Ranging}
}

\newglossaryentry{BU}
{
  name={BU},
    description={Une Business Unit est une unité organisationnelle au sein d'une entreprise, s'articulant autour d'un domaine d'activité donné. Par la création de BU, 
    l'entreprise entend généralement augmenter son chiffre d'affaires et sa marge brute en conférant 
    d'avantage d'autonomie financière et décisionnelle à ces unités stratégiques. On parle également de département, de division ou de domaine fonctionnel}
}

\newglossaryentry{ASD}
{
  name={ASD},
    description={Aero Space Defense est une Business Unit définie par SII}
}

\newglossaryentry{SA}
{
  name={SA \`{a} directoire et conseil de surveillance},
    description={Une Société Anonyme est une forme juridique de société commerciale, dont la dénomination sociale permet entre-autre de protéger l'anonymat de ses actionnaires. 
    SII correspond à une SA à directoire et conseil surveillance.
    Moins répandue qu'une SA à conseil d'administration\cite{Bib_SA_wiki}, ce statut mène à identifier distinctement deux organes de gouvernance dont le directoire représente la partie exécutive. 
    Le conseil de surveillance endosse quant à lui les tâches de nomination et de contrôle du directoire}
}

\newglossaryentry{NTIC}
{
  name={NTIC},
    description={Nouvelles Technologies de L'Information et de la Communication}
}

\newglossaryentry{SI}
{
  name={SI},
    description={Système d'Information}
}

\newglossaryentry{IDF}
{
  name={IDF},
    description={\^{I}le-de-France}
}

\newglossaryentry{ETR}
{
  name={ETR},
    description={Energie Transport Retail est une Business Unit définie par SII}
}

\newglossaryentry{BAM}
{
  name={BAM},
    description={Banque Assurance Mutuelle est une Business Unit définie par SII}
}

\newglossaryentry{POC}
{
  name={POC},
    description={Proof Of Concept (preuve de concept)}
}

\newglossaryentry{QHSE}
{
  name={QHSE},
    description={Qualité Hygiène Sécurité Environnement. Située au niveau de l'agence \^{I}le-de-France, la cellule QHSE veille activement sur le respect de la qualité opérationnelle, tout en s'inscrivant dans une démarche d'amélioration continue. 
    Elle est en charge de la préparation et du maintien des niveaux de certifications de l'entreprise.
    Enfin, elle est impliquée dans la mise en pratique d'une démarche \gls{RSE}, notamment par la prise en compte des impacts environnementaux de l'activité}
}

\newglossaryentry{RSE}
{
  name={RSE},
    description={Responsabilité Sociétale de l'Entreprise}
}

\newglossaryentry{RD}
{
  name={R\&D},
    description={Recherche et Développement}
}

\newglossaryentry{SRT2M}
{
  name={SRT2M},
    description={Système Robotique Tactique Multi-Missions pour la surveillance et l'aide à la prise de décisions dans les milieux à risques. 
    C'est le nom donné au système final intégrant les briques logicielles et physiques issues de mon stage et de celui de M. Chazot}
}

\newglossaryentry{DCL}
{
  name={DCL},
    description={Document de Conception du Logiciel}
}

\newglossaryentry{STBL}
{
  name={STBL},
    description={La Spécification Technique du Besoin Logiciel est un document généralement réalisé par un client pour spécifier le besoin qui justifie la réalisation d'un logiciel}
}

\newglossaryentry{SDK}
{
  name={SDK},
    description={Software Development Kit}
}

\newglossaryentry{IHM}
{
  name={IHM},
    description={Interface Homme-Machine}
}

\newglossaryentry{classe}
{
  name={classe},
    description={La classe d'un objet correspond à un champ textuel qui permet de le qualifier. Ce label est fourni par le réseau neuronal et résulte des opérations de détection et de classification.
    Par exemple, on peut observer les classes "humain", "chaise" ou "ordinateur" pour un réseau neuronal destiné à être employé dans un environnement de bureau}
}

\newglossaryentry{landmarks}
{
  name={points de repères},
    description={Les points de repère sont des caractéristiques observables par le biais de capteurs du robot, lui permettant de se situer dans son environnement et de le carographier. 
    La qualité d'un point de repère s'évalue en fonction des caractéristiques suivantes :
    \begin{itemize}
    \item la facilité de ré-observabilité
    \item la possibilité de discrimines des poinst de repères individuels 
    \item leur quantité dans l'environnement
    \item leur caractère stationnaire
    \end{itemize}
    Un exemple typique est une ligne droite et des coins bien définis tels que des murs délimitant une pièce}
}

\newglossaryentry{KNN}
{
  name={KNN},
    description={K-Nearest Neighbor est un algorithme qui permet de définir la sortie associée à une entrée donnée en considérant les K plus proches voisins connus de cette entrée. 
    La notion de voisinnage est sous-tendue par l'application d'une formule de distance entre l'entrée à traiter et les échantillons connus}
}

\newglossaryentry{XMLRPC}
{
  name={XML-RPC},
    description={XML-RPC est un protocole réseau de type \emph{Remote Procedure Call} permettant l'échange de données et l'appel de méthodes par des exécutables distincts. La définition des données 
    comme des procédures est faite par le biais du langage XML}
}

\newglossaryentry{topics}
{
  name={topics},
    description={Les topics permettent la communication entre les différents noeuds. 
    Un noeud est responsable de la publication des données du topic (commandes de contrôle par exemple) et un noeud distinct souscrit à ce même topic pour les récupérer (par exemple le noeud qui va traîter ses commandes).
    Typiquement, la publication d'un topic se fera en C++ par l'appel à la méthode advertise().
    Parallèlement, la souscription a un topic nécessite l'appel à la méthode subscribe()}
}

\newglossaryentry{services}
{
  name={services},
    description={Les services sont des éléments de communication inter-noeuds qui permettent l'envoi de requêtes et de réponses. 
    La création de services passe par la définition textuelle des types de données attendues dans la requête et la réponse}
}

\newglossaryentry{serveur de parametres}
{
  name={serveur de paramètres},
    description={Le serveur de paramètres est un dictonnaire partagé entre les différents noeuds actifs, accessible grâce aux APIs réseaux. 
    Il n'est pas destiné à supporter une montée en charge importante, de telle sorte que son utilisation doit se limiter à stocker les paramètres d'utilisation des noeuds. 
    Ces derniers pourront récupérer leurs variables de configuration au runtime en invoquant le serveur.
    Le serveur de paramètre utilise le protocole XMLRPC et est impélmenté à l'intérieur du ROS Master}
}

\newglossaryentry{Catkin}
{
  name={Catkin},
    description={Catkin est le nom de la chaîne de compilation utilisée par ROS. 
    Basée sur des macros CMake, catkin automatise un certain nombre de tâches, dont la recherche de packages ROS. 
    Ainsi, Catkin permet la génération d'exécutables, de librairies, d'interfaces exportées ou de scripts auto-générés, 
    chacune de ces cibles devant apartenir et être générées depuis un package contenant le code source requis}
}

\newglossaryentry{SLAM}
{
  name={SLAM},
    description={Le SLAM pour Cartographie et Localisation Simultannées (de l'anglais Simultaneous Localisation And Mapping) est une discipline permettant 
    de conférer à un dispositif mobile la perception de son environnement externe, sous la forme d'une cartographie des points de repères stationnaires, tout comme de ses 
    propres positions et orientations au sein de cet environnement. 
    Le SLAM est mis en \oe{}uvre par le biais de capteurs internes (mesures odométriques ou inertielles) et externes (LIDAR, RADAR, caméra)}
}

\newglossaryentry{Hector SLAM}
{
  name={Hector SLAM},
    description={Hector SLAM est le nom d'un métapackage ROS disponible par le biais du gestionnaire de version Git. 
    Il contient entre-autres un n\oe{}ud de SLAM, nommé hector\_mapping. Hector SLAM est sous licence BSD et a été développé par des chercheurs de l'université de Darmstad (Allemagne) dont 
    Stefan Kohlbrecher et Johannes Meyer. Ses fondements théoriques apparaissent dans une publication datant de 2011\cite{Bib_Hector_SLAM}}
}

\newglossaryentry{IMU}
{
  name={IMU},
    description={Inertial Measurement Unit, désigne des données issues de capteurs inertiels, typiquement présents dans des systèmes robotiques ou des véhicules aériens sous la forme de gyromètres et d'accéléromètres.
    De telles mesures qualifient le roll, le pitch et le yaw (roulis, tangage et lacet en français), à savoir les valeurs des composantes sur les trois degrés de liberté d'un véhicule }
}

\newglossaryentry{GeoTIFF}
{
  name={GeoTIFF},
    description={Tf pour ``The transform library'' est un package ROS qui permet de mettre en relation différents systèmes de coordonnées et de mettre à jour ces relations durant leurs déplacements respectifs}
}

\newglossaryentry{bagfile}
{
  name={bagfile},
    description={Un fichier bag (ou bagfile) est un fichier spécifique à l'écosystème de ROS qui contient des enregistrements de messages sérialisés variés et les timestamps afférents. 
    ROS permet l'enregistrement et la lecture de tels fichiers par le biais d'outils tels que l'exécutable rosbag.
    Ces fichiers sont utilisés à des fins d'analyse ou de simulations de situations pratiques pendant lesquelles des n\oe{}uds émettent et / ou transmettent des données}
}

\newglossaryentry{tf}
{
  name={tf},
    description={Tf, pour \emph{The transform library} est le nom d'un package ROS qui permet de définir les nomenclatures et les transformations entre les systèmes de repères utiles aux n\oe{}uds de calcul}
}

\newglossaryentry{TCP}
{
  name={TCP},
    description={Transport Control Protocol est un protocole de la couche transport qui permet entre-autre la remise d'accusés de réception à l'émetteur (flag ACK) dans un environnement client-serveur. 
    Cette particularité vise à assurer la fiabilité de la communication établie}
}

\newglossaryentry{EKF}
{
  name={EKF},
    description={Extended Kalman Filter}
}

\newglossaryentry{API}
{
  name={API},
    description={Application Programming Interface est une interface donnant accès à un ensemble de méthodes et fonctions normalisées facilitant le développement d'applications}
}

\newglossaryentry{XML}
{
  name={XML},
    description={eXtensible Markup Language est un langage informatique balisé et normalisé, traditionnellement utilisé à des fins de transmission de données via Internet de part son indépendance de toute plateforme}
}